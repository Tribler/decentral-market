\section{Trade Protocol}
\label{protocol}

Tsukiji uses a custom application protocol to communicate between peers.
Traders communicate to each other with messages encoded as JSON objects.
The choice for JSON objects was made because it has a near-equivalent data type in python called a dictionary.
It is also a human-readable format.
Every message has the following additional attributes:
\begin{myitemize}
\item \texttt{id}: Public key of the trader sending the message.
\item \texttt{message-id}: Incremental counter unique across all messages with the same id.
\item \texttt{timestamp}: Time of creation of this message.
\item \texttt{type}: The type of message.
\end{myitemize}
The combination of the \texttt{id} and \texttt{message-id} attributes uniquely identifies each message.

There are 7 message types in total: \texttt{ask}, \texttt{bid}, \texttt{trade}, \texttt{confirm}, \texttt{cancel}, \texttt{greeting}, and \texttt{greeting\_response}. Each type of message is used in a different scenario and holds different attributes. In the following sections, these scenario's are laid out.

\subsubsection{Placing an offer}
In order to place an ask or bid, a message is sent where the type is \texttt{ask} or \texttt{bid} accordingly.
These types of messages have the following additional attributes:
\begin{myitemize}
\item \texttt{price}: The price per single unit.
\item \texttt{quantity}: The amount of items requested to trade.
\item \texttt{timeout}: The time of expiration for this offer.
\end{myitemize}
This type of message is typically sent to the entire network.
For the purpose of this application, the units of \texttt{price} and \texttt{quantity} are left undefined.
In the future, the protocol could be updated with a currency string, such as 'EUR' for euros, 'USD' for dollars, or 'BTC' for bitcoin.
Something similar could be adopted for the \texttt{quantity}, for example, 'kg' for weight, 'm3' for volume, or 'MBit' for data.

\subsubsection{Requesting a trade}
When a trader sees an offer they are interested in, they might respond to such an offer with a trade message.
The type of this message is \texttt{trade}.
This type of message has the following additional attributes:

\begin{myitemize}
\item \texttt{recipient}: The id of the trader who created the offer.
\item \texttt{quantity}: The quantity requested to be traded.
\item \texttt{trade-id}: The message-id of the original offer.
\end{myitemize}

The combination of \texttt{recipient} and \texttt{trade-id} is used to uniquely identify the specific offer for which the trade is requested.
The \texttt{quantity} field may be equal to or less than the quantity stated in the original offer.
Perhaps a trader is interested in an offer, but only can or needs to trade a certain amount that is less than this.

\subsubsection{Accepting a trade}

When a trade request arrives, a trader can either accept or cancel such a request.
In the event that the trader accepts the trade request, they send a confirm message back.
The type of this message is \texttt{confirm}.
This type of message has the following additional attributes:

\begin{myitemize}
\item \texttt{trade-id}: the message-id of the original offer.
\end{myitemize}

This time, there's no need to add a recipient attribute because the combination of the \texttt{id} and \texttt{trade-id} fields uniquely identifies the offer to be traded.
The offer is removed from the internal orderbook of both parties.

\subsubsection{Cancelling an offer}

When a trader decides to cancel one of its offers, e.g., because they want to lower their prices, Tsukiji will respond to any incoming trade requests with a cancel message.
For a more detailed discussion on why this is implemented in this manner, see section \ref{sprint1:cancelling}.
The type of this message is \texttt{cancel}.
This type of message has the following additional attributes:

\begin{myitemize}
\item \texttt{trade-id}: the message-id of the original offer.
\end{myitemize}

Similar to a confirm message, there's no need to add a recipient attribute because the combination of the \texttt{id} and \texttt{trade-id} fields uniquely identifies the offer to be traded.
The offer is removed from the internal orderbook of both parties.

\subsubsection{Peer discovery}

In order to facilitate peer discovery, there needs to be some mechanism for peers to exchange knowledge of other peers. For more information, see sections \ref{sprint2:peerdiscovery} and \ref{sprint2:gossip}.

For a peer to request a list of peers from somebody else, they send a greeting message.
This message has type \texttt{greeting}.
No additional attributes are necessary.

When a trader receives a \texttt{greeting} message, they respond with a greeting\_response.
This message has type \texttt{greeting\_response}.
This type of message has the following additional attributes:
\begin{myitemize}
\item \texttt{peerlist}: a list of peers
\end{myitemize}

When a peer receives such a list of peers, they can add it to their own list of peers.
Now the peers they know about has grown with more traders to directly communicate with.

\subsubsection{Example of a trade}
In order to get a better understanding of the protocol, we will describe an example of a trade. See figure \ref{tradefig} for a graphical depiction of the exchange.
Alice wants to buy 3 items at price 10.
She sends out an \texttt{ask} message to the world.
Bob sees this message and is interested.
He sends a trade request.
Alice, who has received no other trade request up until that point, agrees to the trade and returns a confirm message.

\begin{figure}[H]
  \centering
  \includegraphics[width=\textwidth]{trade}
  \caption{Example of a trade. Alice sends out an offer. Bob replies with a trade request. Alice agrees with the request and replies with a confirm message.}
  \label{tradefig}
\end{figure}