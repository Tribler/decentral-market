\documentclass[]{article}
\usepackage[utf8]{inputenc}
\usepackage{hyperref}

%opening
\title{Tsukiji}
\author{Michael The \and Hugo Reinbergen}

\begin{document}

\maketitle

\begin{abstract}

\end{abstract}

\section{Introduction}
\section{Related Work}
\begin{tabular}{|c|c|}
 \hline
 Document & Contents  \\
 \hline
 \url{https://tinyurl.com/lrqbb2c} & Reputation \\
 \url{https://tinyurl.com/n3v5jsy} & Dispersion \\
 \url{https://bitcoin.org/bitcoin.pdf} & Bitcoin \\
 \url{http://www.weidai.com/bmoney.txt} & b-money \\
 Mailtje van Pouwelse & Credit Based P2P \\
 Book: Computer Networks & DHT, P2P \\
 \hline
\end{tabular} 
\section{Sprint layout}
The development of Tsukiji followed aspects of the Scrum methodology.
\subsection{Sprint 1: Sockets and broadcasting}
The goal of the first sprint was to create a local simulation of the decentral market, using socket connections. The peers had to be able to communicate with each other. The options considered for this were approaching a subset and relaying the information of your subset to other subsets and broadcasting your information to the entire network. The issue that arose with the subset approach was that in a competitive market, someone that is selling an item is not easily motivated to advertise in the name of another seller. This could lead to certain peers that are offering an item for sale, not to relay other offers that would endanger the profits of the peer. This could lead to a disjoint of the network and therefore destroy the communication between certain peers. \\
Broadcasting would not have this issue. Everyone personally takes care of his own advertising and there is no peer between origin and destination that could disrupt the communication. The issue that rises with broadcasting though, is that is scales rather poorly. The increase of data over the network increases exponentially as the amount of users rises. Because of this, using broadcasting in a large scale project is not advised, but since this project is mostly aiming to be a proof of concept and this is also the first iteration, broadcasting would suffice for the time being.\\
\subsubsection{Problems and solutions}
Testing the socket connection showed a couple of issues. First of, whenever a client would disconnect from a peer, the server that it was connected to reported a Broken Pipe error. This however did not stop the server from running but it did add 2 empty string to the collection of received strings. However if the server received an empty string that was sent intentionally, it ignores that message. Therefor it would not hinder the communication if the Broken Pipe errors are suppressed, since the program will not change behaviour.The empty strings that resulted from this error are now filtered out.

\end{document}
